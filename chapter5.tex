\chapter{جمع‌بندي و نتيجه‌گيري و پیشنهادات}
\clearpage
در این فصل، ابتدا به جمع‌بندی و مرور کلی پژوهش انجام‌شده، اهداف، رویکردها و دستاوردهای کلیدی آن می‌پردازیم. سپس، با تکیه بر نتایج به‌دست‌آمده و محدودیت‌های موجود، پیشنهاداتی برای پژوهش‌های آتی در این حوزه ارائه خواهیم داد.

\section{جمع‌بندی}

مسئله‌ی شناسایی فعالیت انسان با استفاده از داده‌های حسگر، یکی از ارکان اساسی در توسعه‌ی سیستم‌های هوشمند مراقبتی، خانه‌های هوشمند و سلامت دیجیتال به شمار می‌رود. با وجود موفقیت‌های چشمگیر روش‌های یادگیری عمیق در این حوزه، وابستگی شدید آن‌ها به حجم زیادی از داده‌های برچسب‌دار، یک چالش اساسی و پرهزینه محسوب می‌شود. فرآیند برچسب‌گذاری داده‌های حسگری نه تنها زمان‌بر و نیازمند نیروی متخصص است، بلکه مدل‌های آموزش‌دیده به روش نظارت‌شده اغلب در تعمیم به افراد جدید یا شرایط متغیر دنیای واقعی، با افت عملکرد مواجه می‌شوند.

این پژوهش با هدف غلبه بر این محدودیت‌ها، به بررسی و پیاده‌سازی یک چارچوب یادگیری خودنظارتی برای استخراج بازنمایی‌های غنی و تعمیم‌پذیر از سیگنال‌های مربوط به فعالیت انسان پرداخت. معماری پایه، که بر دو رمزگذار مجزا برای تحلیل همزمان داده‌ها در حوزه‌ی زمان (سیگنال خام) و حوزه‌ی زمان-فرکانس (اسکالوگرام) استوار بود، به عنوان نقطه شروع انتخاب شد. با این حال، با تحلیل این معماری، دو حوزه اصلی برای بهبود شناسایی و مورد هدف قرار گرفت.

نوآوری نخست، جایگزینی الگوریتم یادگیری تباینی \lr{SimCLR} با روش پیشرفته‌تر و مبتنی بر خوشه‌بندی \lr{SwAV} بود. این الگوریتم با بهره‌گیری از یک سازوکار پیش‌بینی تعویض‌شده و خوشه‌بندی برخط، مدل را به یادگیری بازنمایی‌هایی پایدارتر و متمایزتر وادار می‌کند که نسبت به تغییرات و تبدیلات داده‌افزایی، نامتغیر هستند. نوآوری دوم، بازنگری و بهبود راهبرد داده‌افزایی برای اسکالوگرام‌ها بود. به جای استفاده از تبدیلات اقتباس‌شده از حوزه‌ی بینایی کامپیوتر که فاقد معنای فیزیکی روشنی برای داده‌های حسگری هستند، مجموعه‌ای از تبدیلات معنادارتر و سازگارتر با ماهیت سیگنال‌ها به کار گرفته شد.

برای ارزیابی عملکرد روش پیشنهادی، دو سناریوی آزمایشی جامع طراحی گردید. در سناریوی درون‌مجموعه‌ای، که پیش‌آموزش و تنظیم دقیق بر روی مجموعه داده‌ی واحد \lr{HAPT} انجام شد، نتایج نشان داد که رویکرد خودنظارتی پیشنهادی، به ویژه در شرایط کمبود داده‌های برچسب‌دار، برتری چشمگیری نسبت به یادگیری کاملاً نظارت‌شده دارد و در اکثر موارد، عملکردی بهتر از روش پایه ارائه می‌دهد. مهم‌تر از آن، در سناریوی چالش‌برانگیزتر بین‌مجموعه‌ای که به منظور سنجش قدرت یادگیری انتقالی طراحی شده بود (پیش‌آموزش بر روی مجموعه داده‌ی بزرگ \lr{MobiAct} و تنظیم دقیق بر روی \lr{HAPT})، برتری مدل پیشنهادی کاملا مشهود بود. در این سناریو که شبیه‌ساز شرایط واقعی‌تر با وجود تغییر دامنه بین داده‌های آموزشی و ارزیابی است، مدل پیشنهادی پایداری و قابلیت تعمیم بالاتری از خود به نمایش گذاشت. این یافته‌ی کلیدی نشان می‌دهد که بازنمایی‌های آموخته‌شده توسط روش پیشنهادی، عمومی‌تر، بنیادی‌تر و در برابر تغییر دامنه مقاوم‌تر هستند.این یافته‌ی کلیدی نشان می‌دهد که بازنمایی‌های آموخته‌شده توسط روش پیشنهادی، عمومی‌تر، بنیادی‌تر و در برابر تغییر دامنه مقاوم‌تر هستند.

در مجموع، نتایج این پژوهش تأیید می‌کند که ترکیب الگوریتم یادگیری خودنظارتی \lr{SwAV} با راهبردهای داده‌افزایی متناسب، منجر به توسعه‌ی مدلی قدرتمندتر و قابل‌اطمینان‌تر برای شناسایی فعالیت انسان می‌شود که گامی مؤثر در جهت کاهش وابستگی به داده‌های برچسب‌دار و افزایش کارایی مدل‌ها در کاربردهای دنیای واقعی است.

\section{کارهای آتی}

با توجه به نتایج امیدوارکننده‌ی این پژوهش، مسیرهای متعددی برای توسعه و بهبود کارهای آتی قابل تصور است که در ادامه به برخی از مهم‌ترین آن‌ها اشاره می‌شود:
\begin{itemize}
    \item \textbf{به‌کارگیری معماری‌های پیشرفته‌تر:} رمزگذار‌های استفاده‌شده در این پژوهش مبتنی بر شبکه‌های پیچشی نسبتا ساده بودند. می‌توان با بهره‌گیری از معماری‌های پیشرفته‌تر مانند مدل مبدل که توانایی بالایی در مدل‌سازی وابستگی‌های طولانی‌مدت دارند، یا معماری‌های ترکیبی مانند \lr{CNN-LSTM}، بازنمایی‌های دقیق‌تری از الگوهای زمانی پیچیده استخراج کرد.
    \item \textbf{راهبردهای ادغام چندوجهی:} در این پژوهش از راهبرد ادغام دیرهنگام برای ترکیب خروجی دو رمزگذار استفاده شد. تحقیق بر روی راهبردهای ادغام زودهنگام یا میانی، که در آن‌ها بازنمایی‌های حوزه‌ی زمان و زمان-فرکانس در لایه‌های پایین‌تر با یکدیگر ترکیب می‌شوند، می‌تواند به مدل اجازه دهد تا وابستگی‌های متقابل میان این دو حوزه را بهتر بیاموزد.
    \item \textbf{شخصی‌سازی و یادگیری مداوم:} مدل فعلی به صورت عمومی آموزش دیده است. یک مسیر تحقیقاتی ارزشمند، توسعه‌ی روش‌هایی برای شخصی‌سازی سریع مدل برای یک کاربر جدید با استفاده از حجم بسیار کمی داده یا تطبیق مداوم مدل با تغییرات تدریجی در الگوهای حرکتی یک فرد در طول زمان (یادگیری برخط) است.
    \item \textbf{ارزیابی در سناریوهای پیچیده‌تر:} عملکرد مدل را می‌توان در مجموعه داده‌های پیچیده‌تر و نزدیک‌تر به واقعیت ارزیابی کرد؛ مجموعه داده‌هایی که شامل فعالیت‌های همزمان و هم‌پوشان، تعداد بیشتری از دسته‌های فعالیت، یا داده‌های حاصل از منابع متنوع‌تری (مانند حسگرهای محیطی در کنار حسگرهای پوشیدنی) هستند.
\end{itemize}
