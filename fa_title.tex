%% -!TEX root = AUTthesis.tex
% در این فایل، عنوان پایان‌نامه، مشخصات خود، متن تقدیمی‌، ستایش، سپاس‌گزاری و چکیده پایان‌نامه را به فارسی، وارد کنید.
% توجه داشته باشید که جدول حاوی مشخصات پروژه/پایان‌نامه/رساله و همچنین، مشخصات داخل آن، به طور خودکار، درج می‌شود.
%%%%%%%%%%%%%%%%%%%%%%%%%%%%%%%%%%%%
% دانشکده، آموزشکده و یا پژوهشکده  خود را وارد کنید
\faculty{دانشکده مهندسی کامپیوتر}
% گرایش و گروه آموزشی خود را وارد کنید
\department{گرایش هوش مصنوعی و رباتیکز}
% عنوان پایان‌نامه را وارد کنید
\fatitle{شناسایی فعالیت‌های انسانی در محیط‌های هوشمند با
\\[.75 cm]
استفاده از یادگیری خودنظارتی}
% نام استاد(ان) راهنما را وارد کنید
\firstsupervisor{دکتر احسان ناظرفرد}
%\secondsupervisor{استاد راهنمای دوم}
% نام استاد(دان) مشاور را وارد کنید. چنانچه استاد مشاور ندارید، دستور پایین را غیرفعال کنید.
% \firstadvisor{نام کامل استاد مشاور}
%\secondadvisor{استاد مشاور دوم}
% نام نویسنده را وارد کنید
\name{اردلان }
% نام خانوادگی نویسنده را وارد کنید
\surname{نهاوندی فرد}
%%%%%%%%%%%%%%%%%%%%%%%%%%%%%%%%%%
\thesisdate{شهریور 1404}

% چکیده پایان‌نامه را وارد کنید
\fa-abstract{
با گسترش روزافزون محیط‌های هوشمند و استفاده از حسگرهای مختلف مانند تلفن همراه، نیاز به سیستم‌هایی که بتوانند به‌صورت خودکار و دقیق فعالیت‌های انسانی را تشخیص دهند، افزایش یافته است. یکی از چالش‌های اصلی در این حوزه، وابستگی شدید مدل‌های یادگیری ماشین به داده‌های برچسب‌خورده است که جمع‌آوری آن‌ها در مقیاس بزرگ، پرهزینه و زمان‌بر است. این موضوع، ضرورت بهره‌گیری از روش‌هایی را مطرح می‌سازد که بدون نیاز به برچسب‌گذاری دستی، بتوانند نمایش‌های مفهومی و قابل‌انتقال از داده‌های حسگر استخراج کنند. در این پژوهش، یک چارچوب یادگیری خودنظارتی طراحی شده که با بهره‌گیری از ترکیب دیدگاه‌های زمانی و فرکانسی، تلاش می‌کند نمایش‌هایی باکیفیت و عمومی از داده‌های خام فعالیت انسانی استخراج نماید. چارچوب پیشنهادی با هدف بهبود کیفیت بازنمایی داده، کاهش نیاز به داده‌های برچسب‌خورده، و افزایش قابلیت تعمیم‌پذیری مدل، در دو سناریوی مختلف مورد ارزیابی قرار گرفته است: نخست، آموزش و ارزیابی در یک محیط یکسان؛ و دوم، آموزش در یک محیط و ارزیابی در محیطی متفاوت، با هدف سنجش توانایی انتقال دانش و تعمیم‌پذیری مدل.
نتایج حاصل از ارزیابی‌ها نشان می‌دهد که چارچوب ارائه‌شده در سناریوی آموزش مستقیم بر روی مجموعه داده \lr{HAPT} به \textbf{امتیاز \lr{F1} \%۳.۹۲} (با \textbf{\%۹.۰ بهبود} نسبت به مدل پایه) و در سناریوی یادگیری انتقالی از مجموعه داده \lr{MobiAct} به \lr{HAPT} به \textbf{امتیاز \lr{F1} \%۲.۹۰} (با \textbf{\%۵.۱ بهبود} نسبت به مدل پایه) دست یافته است.
این چارچوب توانسته با بهره‌گیری از ترکیب اطلاعات زمانی و فرکانسی، بازنمایی‌هایی استخراج کند که منجر به بهبود دقت در تشخیص فعالیت‌های انسانی شده‌اند. در مجموع، روش پیشنهادی گامی مؤثر در جهت کاهش وابستگی به داده‌های برچسب‌خورده و توسعه‌ی مدل‌های تعمیم‌پذیر برای کاربرد در محیط‌های متنوع و واقعی برداشته است.
}


% کلمات کلیدی پایان‌نامه را وارد کنید
\keywords{شناسایی فعالیت انسان، یادگیری خودنظارتی، تبدیل موجک، یادگیری تباینی، یادگیری انتقالی}



\AUTtitle
%%%%%%%%%%%%%%%%%%%%%%%%%%%%%%%%%%
\vspace*{7cm}
\thispagestyle{empty}
\begin{center}
\includegraphics[height=5cm,width=12cm]{Images/besm.jpg}
\end{center}